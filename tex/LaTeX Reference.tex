
%----------------------------------------------------------------------------
% Equation code. Equations are referencable through their number.
%----------------------------------------------------------------------------

\begin{equation}
x = h_{d}/m
\end{equation}
\begin{equation}
y_{n} = y_{n - 1} + (x_{n} - x_{n - 1})\tan(\theta_{n})
\end{equation}

%----------------------------------------------------------------------------
% Figure code. Here's a breakdown of what makes this document friendly:
%
% [H] forces the image to display exactly where it is placed within a document.
% [width=1.0\textwidth] limits figure size to the width of the text area.
% The caption should describe the figure clearly and concisely.
% The label should follow a convention making it easy to reference.
%----------------------------------------------------------------------------

\begin{figure}[H]
\centering
\includegraphics[width=1.0\textwidth]{\imgpath/PICTURE OR FIGURE HERE}
\caption{}
\label{}
\end{figure}

%----------------------------------------------------------------------------
% Table code. tableizer.m in Matlab Utilities allows for easy conversion of
% a lot of MATLAB code into LaTeX tabular format.
%----------------------------------------------------------------------------

\begin{tabular}{|c|c|c|}
  \hline
  L = 130 m. & h = 2.5 m & $\theta_{o}$ = $20^{o}$ \\
  \hline
  Values Calculated: & $K_{r}$ & $\theta$  \\ 
  \hline
  Hand Calculation & 0.967 & 2.34\\
  \hline
  Function & 0.9698 & 2.357 \\
  \hline
\end{tabular}